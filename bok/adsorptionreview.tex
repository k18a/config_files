


#+TITLE: A Critical Review of Experimental and Theoretical Assumptions Involved in Manometric Adsorption Characterization of Gas Shales and their Net Effects on Reserve Estimation and Production Forecasts

* Introduction
- nearly 20-80% of the total gas in gas shales is stored in the adsorbed state 
- lack of reproducibility of adsorption isotherms measured in different laboratories is a significant roadblock to shale reservoir characterization
- whilst we concede that the highly heterogeneous nature of shales are at least partly responsible for this, significant improvements could be made through diligent control and reporting of experimental parameters
- this study will review our current understanding of manometric adsorption characterization of shales, and provide a critical analysis on the effect of the assumptions involved in the process on reservoir characterization, through an one dimensional reservoir simulation

* Methodology
- a method of analysis after cite:Rouquerol1994 is adopted
- for all samples, rate of sorption after 2 hours was less than -----
- where experimental equilibrium amount was singnificantly different from modelled equilibrium values, the modelled equilibrium values were chosen to fit adsorption isotherms
- the model with the least mean R squared value accross all pressure steps of a given isotherm was chosen for this purpose

* Results
  
** 

* Discussions

* Assumptions on Void Volume
- a specific volume of 0.5 cc/g might be too high for most gas shales, whose typical densities range between ----

* Assumptions on leak rates
- home-made high pressure adsorption rigs usually have a small leak
- due to the long equilibrium times required for shale adsorption, even small leaks may have a significant effect in adorption calculations; these must be accounted for
- several studies have incorrectly assumed a constant leak rate for all adsorption steps
- leak rate is pressure dependent and must be accounted for using Poiseuille's law, which characterizes fluid flow behaviour in very narrow openings
- hence it's sufficient to measure leak resistance at one sufficiently high pressure, to be able to account for it for the entire adsorption isotherm

- it was noted that leak rates were not constant with time - they varied between various experiments based on the efficiency of the compression fittings used
- it is preferable to characterise leak rates just before starting the experiment

  
* Assumptions on Kinetic Models
- it was noted that most of the sorption occured within a few fractions of a seconds
- kinetic models were fit using a modified least squares approach, providing more weight to data points with smaller time scales
  - possibly take R^2 * t as the error instead of R^2?
- 
- 

* Assumptions on Equation of State
- 

* Assumptions on Equilibrium Conditions
- equilibrium was determined based on a rolling regression fit between amount adsorbed against time with a window of 200 (~30 mins)
- when the slope of the regression line (rate of sorption) fell below 5e-8 mmol/g/sec, it was assumed that equilibrium has been achieved


