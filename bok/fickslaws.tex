% Created 2018-10-11 Thu 19:32
% Intended LaTeX compiler: pdflatex
\documentclass[11pt]{article}
\usepackage[utf8]{inputenc}
\usepackage[T1]{fontenc}
\usepackage{graphicx}
\usepackage{grffile}
\usepackage{longtable}
\usepackage{wrapfig}
\usepackage{rotating}
\usepackage[normalem]{ulem}
\usepackage{amsmath}
\usepackage{textcomp}
\usepackage{amssymb}
\usepackage{capt-of}
\usepackage{hyperref}
\date{\today}
\title{Fick's Laws}
\hypersetup{
 pdfauthor={},
 pdftitle={Fick's Laws},
 pdfkeywords={},
 pdfsubject={},
 pdfcreator={Emacs 26.1 (Org mode 9.1.13)}, 
 pdflang={English}}
\begin{document}

\maketitle
\tableofcontents


\section{Fick's First Law}
\label{sec:org1e13411}
\begin{itemize}
\item an analogy with the \href{fourierslaws.org}{Fourier's laws} for heat transfer
\item rate of transfer of the diffusing substance through unit area of a section is proportional to the concentration gradient measured normal to the section
\end{itemize}
\begin{equation}
F = D \frac{\partial C}{\partial x}
\end{equation}
\begin{itemize}
\item defines mass flux with respect to distance
\item first law for cartesian co-ordinates
\begin{itemize}
\item $$J = - A * D * \frac{\partial C}{\partial z}$$
\item J is flux per unit area
\item A is area accross which diffusion occurs
\item C is concentration
\item z is distance
\item D is diffusion coefficient
\end{itemize}
\end{itemize}
\section{Fick's Second Law}
\label{sec:org3bee403}
\subsection{Derivation}
\label{sec:org38a6280}
\begin{itemize}
\item let's assume an isotropic medium
\item consider the following volume - a rectangular parallepiped with sides 2dx, 2dy, and 2dz
\item let concentration at the centre be C

    K+---------------------------+K'
    \emph{|                          /|
2dz} |                         / |
  /  |                        /  |
L+---------------------------+L' |---------> 
\begin{center}
\begin{tabular}{llll}
 &  &  & 4dydz(Fx+dFx/dxdx)\\
\end{tabular}
\end{center}
----------->|   |                       |   |
\end{itemize}
4dydz(Fx+dFx/dx dx)| C |                       |   |
                2dy|  J+-----------------------|---+J'
\begin{center}
\begin{tabular}{}
\end{tabular}
\end{center}
I+---------------------------+I'

\begin{itemize}
\item rate at which diffusing substance enters the control volume is given as
\end{itemize}
\begin{equation}
4 dy dz (F_x - \frac{\partial F_x}{\partial x} dx)
\end{equation}
\begin{itemize}
\item rate at which diffusing substance exits the control volume is given as
\end{itemize}
\begin{equation}
4 dy dz (F_x + \frac{\partial F_x}{\partial x} dx)
\end{equation}
\begin{itemize}
\item performing mass balance, net accumulation in the x direction is given as
\end{itemize}
\begin{equation}
-8 dx dy dz \frac{\partial F_x}{\partial x}
\end{equation}
\begin{itemize}
\item similarly in y and z directions
\end{itemize}
\begin{equation}
-8 dx dy dz \frac{\partial F_y}{\partial y}
\end{equation}
\begin{equation}
-8 dx dy dz \frac{\partial F_z}{\partial z}
\end{equation}
\begin{itemize}
\item since average concentration is given as C, the rate at which diffusing amount changes is given as volume * concentration difference
\end{itemize}
\begin{equation}
8 dx dy dz \frac{\partial C}{\partial t}
\end{equation}
\begin{itemize}
\item equating the above equations

\item a more general conservation equation that gives change in concentration with time
\begin{itemize}
\item $$\frac{\partial C}{\partial t} = D * \frac{\partial^2 C}{\partial r^2} = D (\frac{\partial ^2 C}{\partial z ^2} + \frac{1}{A} + \frac{\partial A}{\partial z} \frac{\partial C}{\partial z})$$
\item t is time
\end{itemize}
\item For a non-linear isotherm \cite{Clarkson1999}
\begin{itemize}
\item $$\frac{\partial}{\partial z} (D {\frac{\partial C}{\partial z}}) = \frac{\partial C}{\partial t}$$
\end{itemize}
\end{itemize}
\section{Anisotropic Media}
\label{sec:org714a654}
\section{Analogy with Heat Flow}
\label{sec:org93b01a6}
\section{Diffusion Constant}
\label{sec:orgad150e5}
\begin{itemize}
\item much less sensitive to temperature than the rate constant of a chemical reaction or other phenomena
\item also much less sensitive to the solute being studied, diffusion constant in a given system for a variety of solutes often fall within a factor of 10
\item also much less sensitive to solute concentration, although there are exceptions
\begin{itemize}
\item smaller solutes may show concentration dependent diffusion
\end{itemize}
\end{itemize}
\end{document}
