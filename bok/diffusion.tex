% Created 2018-10-30 Tue 17:35
% Intended LaTeX compiler: pdflatex
\documentclass[11pt]{article}
\usepackage[utf8]{inputenc}
\usepackage[T1]{fontenc}
\usepackage{graphicx}
\usepackage{grffile}
\usepackage{longtable}
\usepackage{wrapfig}
\usepackage{rotating}
\usepackage[normalem]{ulem}
\usepackage{amsmath}
\usepackage{textcomp}
\usepackage{amssymb}
\usepackage{capt-of}
\usepackage{hyperref}
\date{\today}
\title{Diffusion}
\hypersetup{
 pdfauthor={},
 pdftitle={Diffusion},
 pdfkeywords={},
 pdfsubject={},
 pdfcreator={Emacs 26.1 (Org mode 9.1.13)}, 
 pdflang={English}}
\begin{document}

\maketitle
\tableofcontents

\begin{equation}
i_9
\end{equation}

\begin{itemize}
\item process by which matter is transported from one part of a system to another as a result of random molecular motions
\item caused by brownian motion of particles
\item approximate diffusion in
\begin{itemize}
\item gases is 5 cm/min
\item liquids is 0.05 cm/min
\item solids is 0.00001 cm/min
\end{itemize}
\item often the rate determining step in a series of physical processes
\end{itemize}

Diffusion may either be characterized using 

\begin{center}
\begin{tabular}{ll}
rate constant & fick's laws\\
lumped parameter model & distributed parameter model\\
\(flux = k * \Delta(concentration)\) & \(flux = D * (\frac{\Delta(concentration)}{length}\)\\
k is the mass transfer coefficient & D is the diffusion coefficient\\
k is analogous to reciprocal of resistance & D is analogous to reciprocal of resistivity\\
\(voltage = resistance * current\) & \(\frac{voltage}{length}= resistivity * current density\)\\
more empirical & more theoretical\\
\end{tabular}
\end{center}

\begin{itemize}
\item models used to fit experimental kinetic data often fail to consider if the boundary conditions assumed by the model are valid to the given experimental conditions \cite{Qiu2009}
\end{itemize}
\end{document}
