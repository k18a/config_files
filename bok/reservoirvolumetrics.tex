% Created 2018-08-26 Sun 23:39
% Intended LaTeX compiler: pdflatex
\documentclass[11pt]{article}
\usepackage[utf8]{inputenc}
\usepackage[T1]{fontenc}
\usepackage{graphicx}
\usepackage{grffile}
\usepackage{longtable}
\usepackage{wrapfig}
\usepackage{rotating}
\usepackage[normalem]{ulem}
\usepackage{amsmath}
\usepackage{textcomp}
\usepackage{amssymb}
\usepackage{capt-of}
\usepackage{hyperref}
\date{\today}
\title{Reservoir Volumetrics}
\hypersetup{
 pdfauthor={},
 pdftitle={Reservoir Volumetrics},
 pdfkeywords={},
 pdfsubject={},
 pdfcreator={Emacs 26.1 (Org mode 9.1.13)}, 
 pdflang={English}}
\begin{document}

\maketitle
\tableofcontents

\begin{itemize}
\item reservoir volumetric analysis is used to determine volume from static information that remains constant over a period of time
\item Original Oil in Place (OOIP)
\begin{itemize}
\item \(N = \frac{7758 \phi A h_o S_{oi}}{B_{oi}}\)
\begin{itemize}
\item N is original oil in place (STB)
\item \(\phi\) is reservoir porosity
\item A is reservoir area (acres)
\item h\(_{\text{o}}\) is net thickness of oil zone (feet)
\item \(S_{oi}\) is initial reservoir oil saturation
\item \(B_{oi}\) is oil formation volume factor (RB/STB)
\end{itemize}
\end{itemize}
\item Original Free Gas in Place
\begin{itemize}
\item \(G = \frac{7758 \phi A h_g S_{gi}}{B_{gi}}\)
\begin{itemize}
\item G is original gas in place (SCF)
\item \(\phi\) is reservoir porosity
\item A is reservoir area (acres)
\item h\(_{\text{g}}\) is net thickness of gas zone (feet)
\item \(S_{gi}\) is initial reservoir gas saturation
\item \(B_{gi}\) is gas formation volume factor (RB/STB)
\end{itemize}
\end{itemize}
\item the reservoir is modeled by subdividing the reservoir volume into an array or grid of smaller volume elements - gridblock, cell, or node
\item bulk volume V\(_{\text{b}}\) of each gridblock defined in a cartesian co-ordinate system \{x,y,z\} is calculated from the gross thickness \(\Delta\) z = h of each gridblock and the gridblock lengths \(\Delta\) x and \(\Delta\) y
\begin{itemize}
\item \(V_B = h \Delta x \Delta y\)
\end{itemize}
\item gridblock pore volume is calculated as
\begin{itemize}
\item \(V_P = \phi \eta V_B = \phi \eta h \Delta x \Delta y = \phi h_{net} \Delta x \Delta y\)
\item where \(\phi\) is porosity and \(\eta\) is net-to-gross ratio and \(h_{net}\) is net thickness
\end{itemize}
\item volume phase is pore volume times the saturation
\begin{itemize}
\item \(V_l = S_l * V_P = S_l \phi h_{net} \Delta x \Delta y\)
\end{itemize}
\item this estimate from a bottom-up approach can be validated based on volumetrics calculated from other sources such as a material balance study or a computer mapping package
\end{itemize}
\section{\href{materialbalance.org}{Material Balance}}
\label{sec:org8b6b8c1}
\begin{itemize}
\item provide an independent method of estimating the volume of oil, water, and gas in a reservoir for comparison with volumetric estimates
\item magnitues of various factors in a material balance equation indicates the relative contribution of different drive mechanisms at work in the reservoir
\item can be used to predict future reservoir performance and aid in estimating cumulative recovery efficiency
\item based on the law of conservation of mass
\end{itemize}
\subsection{Volumetric Material Balance By Drake}
\label{sec:org62b63de}
\subsubsection{Oil Reservoir with a Gas Cap and an Aquifer}
\label{sec:org35dc2ad}
\begin{center}
\begin{tabular}{lll}
 & term & significance\\
 & \(N D_o\) & change in volume of initial oil and associated gas\\
+ & \(N D_{go}\) & change in volume of free gas\\
+ & \(N (D_w + D_{gw})\) & change in volume of initial connate water\\
+ & \(N D_r\) & change in formation pore volume\\
= & \(N_p B_o\) & cumulative oil production\\
+ & \(N_p R_{so} B_g\) & cumulative gas produced in solution with oil\\
+ & \(G_{ps} B_g\) & cumulative soltion gas produced as evolved gas\\
+ & \(G_{pc} B_{gc}\) & cumulative gas cap gas production\\
+ & \(G_i B_g'\) & cumulative gas injection\\
+ & \(W_e B_w\) & cumulative water influx\\
+ & \(W_l B_w\) & cumulative water injection\\
+ & \(W_p B_w\) & cumulative water production\\
\end{tabular}
\end{center}
\begin{itemize}
\item where
\begin{itemize}
\item \(D_o = (B_t - B_{ti})\)
\item \(D_{go} = N m B_{ti} (\frac{B_{gc} - B_{gi}}{B _{gi}})\)
\item \(D_w = \frac{B_{ti} S_{wio}}{1-S_{wio}} (\frac{B_{tw} - B_{twi}}{B _{twi}})\)
\item \(D_{gw} = \frac{m B_{ti} S_{wig}}{1-S_{wig}} (\frac{B_{tw} - B_{twi}}{B _{twi}})\)
\item \(D_r = (\frac{1}{1-S_{wio}} + \frac{m}{1-S_{wig}) B_{ti} c_f \Delta P\)
\end{itemize}
\item $$N[D_o + D_{go} + D_w + D_{gw} + D_r = N_p B_o - N_p R_{so} B_g + [G_{ps} B_g + G_{pc} B_{gc} - G_i B_g'] - (W_e + W_l - W_p) B_w$$
\begin{itemize}
\item the left-hand-side term accounts for changes in volume
\item terms on the right hand side account for fluid production and injection
\end{itemize}
\end{itemize}
\subsubsection{Gas Reservoir}
\label{sec:org2d41eeb}
\begin{itemize}
\item \(G B_{gi} = N m B _{ti}\)
\item substituting into the oil reservoir material balance equation and accounting for the fact that water compressibility and formation compressibility are negligible compared to gas compressibility
\item $$G B_{gi} (\frac{B_{gc} - B_{gi}}{B_{gi}}) = [G _{pc} B _{gc} - G _t B _g'] - (W _e + W _i - W _p) B _w$$
\end{itemize}
\section{\href{materialtimederivative.org}{Material form of Mass Conservation}}
\label{sec:orgb968cfa}
\begin{itemize}
\item consider a control volume, whose shape and volume can change over time, but is of fixed mass and material
\item reference configuration is at time t=0
\item \(\omega_{\text{0}}\) represents all particles in the control volume at time t=0
\item the control volume changes over time
\item X represents a mapping function that maps the position of a particle x from time 0 to current time t
\item since x(t=0) is represented by X (see \href{materialtimederivative.org}{Material Time Derivative}), X maps co-ordinates from Lagrangian to Eulerian co-ordinates
\item X is a function of X and t
\end{itemize}
x=\Chi(X,t)
\begin{itemize}
\item based on our definition, the mass of the particles in the reference configuration equals mass of particles in current configuration
\end{itemize}
m(\omega_0) = m(\omega)
\begin{itemize}
\item m represents mass of particles
\item in differential form
\end{itemize}
dm(X) = dm(x)
\begin{itemize}
\item X represents position vector in Lagrangian coordinates, and x represents position vector in Eulerian co-ordinates
\item dm is differntial mass
\item mass is volume times density
\end{itemize}
\rho_0(X) \phi(X) dV_0 = \rho(x,t) \phi(x,t) dV 
\begin{itemize}
\item \(\rho\) is density
\item \(\phi\) is rock porosity
\item dV is differential volume
\item integrating this
\end{itemize}
\int \int \int _{\omega_0} \rho_0(X) \phi_0(X) dX_1 dX_2 dX_3= \int \int \int _{\omega} \rho(x,t) \phi(x,t) dx_1 dx_2 dx_3
\begin{itemize}
\item we can use a Jacobian determinant to take Eulerian co-ordinates to Lagrangian co-ordinates
\end{itemize}
\int \int \int _{\omega_0} \rho_0(X) \phi(X) dX_1 dX_2 dX_3= \int \int \int _{\omega} \rho(x,t) \phi(x,t) J dX_1 dX_2 dX_3
\begin{itemize}
\item J is the Jacobian determinant which converts from x to X, given as J = det(F)
\end{itemize}
J = 
\[
\begin{matrix}
 \frac{\partial x_1}{\partial X_1} & \frac{\partial x_1}{\partial X_2} & \frac{\partial x_1}{\partial X_3} \\
 \frac{\partial x_2}{\partial X_1} & \frac{\partial x_2}{\partial X_2} & \frac{\partial x_2}{\partial X_3} \\
 \frac{\partial x_3}{\partial X_1} & \frac{\partial x_3}{\partial X_2} & \frac{\partial x_3}{\partial X_3} \\
\end{matrix}
\]
\begin{itemize}
\item representing
\end{itemize}
F_{ij} = \frac{\partial x_i}{\partial X_j}
\begin{itemize}
\item the Jacobian is given as
\end{itemize}
J = 
\[
\begin{matrix}
 F_{11} & F_{12} & F_{13} \\
 F_{21} & F_{22} & F_{23} \\
 F_{31} & F_{32} & F_{33} \\
\end{matrix}
\]
\begin{itemize}
\item in simplified notation
\end{itemize}
\int _{\omega_0} \rho_0(X) \phi_0(X) dV_0 = \int _{\omega} \rho(x,t) \phi(x,t) J dV_0
\begin{itemize}
\item the statement of mass conservation is thus given as
\end{itemize}
\rho_0(X) \phi_0(X) = \rho(x,t) \phi(x,t) J
\begin{itemize}
\item this is equivalent to other forms of \href{materialbalance.org}{material balance}
\item to get to a more familiar form, we need to take the material time derivative on both sides of the equation
\end{itemize}
\frac{D}{Dt} (\rho _0 \phi _0) = \frac{D}{Dt} (\rho \phi J)
\begin{itemize}
\item left hand side is constant
\end{itemize}
0 = \frac{D}{Dt} (\rho \phi J)
= J \frac{D}{Dt} (\rho \phi) + \rho \phi \frac{D}{Dt} J 
= J [\frac{\partial (\rho \phi)}{\partial t} + \nabla (\rho \phi) * v] + \rho \phi \frac{D}{Dt} J
\begin{itemize}
\item let's figure out the second term
\item applying chain rule, since F is a function of time
\end{itemize}
\frac{D}{Dt} J = \frac{D}{Dt} (det F)
= \frac{\partial det F}{\partial F_{ij}} \frac{\partial F_{ij}}{\partial t}
\begin{itemize}
\item partial of the determinant of a matrix with respect to its components is equal to the determinant of the matrix times the transpose inverse of the matrix
\end{itemize}
\end{document}
